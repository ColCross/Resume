%%%%%%%%%%%%%%%%%%%%%%%%%%%%%%%%%%%%%%%
% Deedy - One Page Two Column Resume
% LaTeX Template
% Version 1.1 (30/4/2014)
%
% Original author:
% Debarghya Das (http://debarghyadas.com)
%
% Original repository:
% https://github.com/deedydas/Deedy-Resume
%
% IMPORTANT: THIS TEMPLATE NEEDS TO BE COMPILED WITH XeLaTeX
%
% This template uses several fonts not included with Windows/Linux by
% default. If you get compilation errors saying a font is missing, find the line
% on which the font is used and either change it to a font included with your
% operating system or comment the line out to use the default font.
% 
%%%%%%%%%%%%%%%%%%%%%%%%%%%%%%%%%%%%%%
% 
% TODO:
% 1. Integrate biber/bibtex for article citation under publications.
% 2. Figure out a smoother way for the document to flow onto the next page.
% 3. Add styling information for a "Projects/Hacks" section.
% 4. Add location/address information
% 5. Merge OpenFont and MacFonts as a single sty with options.
% 
%%%%%%%%%%%%%%%%%%%%%%%%%%%%%%%%%%%%%%
%
% CHANGELOG:
% v1.1:
% 1. Fixed several compilation bugs with \renewcommand
% 2. Got Open-source fonts (Windows/Linux support)
% 3. Added Last Updated
% 4. Move Title styling into .sty
% 5. Commented .sty file.
%
%%%%%%%%%%%%%%%%%%%%%%%%%%%%%%%%%%%%%%%
%
% Known Issues:
% 1. Overflows onto second page if any column's contents are more than the
% vertical limit
% 2. Hacky space on the first bullet point on the second column.
%
%%%%%%%%%%%%%%%%%%%%%%%%%%%%%%%%%%%%%%

\documentclass[]{deedy-resume-openfont}


\begin{document}

%%%%%%%%%%%%%%%%%%%%%%%%%%%%%%%%%%%%%%
%
%     LAST UPDATED DATE
%
%%%%%%%%%%%%%%%%%%%%%%%%%%%%%%%%%%%%%%
%\lastupdated

%%%%%%%%%%%%%%%%%%%%%%%%%%%%%%%%%%%%%%
%
%     TITLE NAME
%
%%%%%%%%%%%%%%%%%%%%%%%%%%%%%%%%%%%%%%


\namesection{}{Colten Cross}{
\href{mailto:colten@cwcross.com}{colten@cwcross.com} | (762) 207-0132}

\vspace{5mm}

%%%%%%%%%%%%%%%%%%%%%%%%%%%%%%%%%%%%%%
%
%     SUMMARY
%
%%%%%%%%%%%%%%%%%%%%%%%%%%%%%%%%%%%%%%

\section{SUMMARY}
\begin{flushleft}
Full Stack Developer leveraging over three years of experience in designing and building fault tolerant and scalable web services. Skilled at liaising between technical and functional team members to solve complex problems with innovative solutions. Thrives working in a close team and agile environment.
\end{flushleft}
\vspace*{-3mm}

\begin{minipage}[t]{0.33\textwidth}

%%%%%%%%%%%%%%%%%%%%%%%%%%%%%%%%%%%%%%
%
%     COLUMN ONE
%
%%%%%%%%%%%%%%%%%%%%%%%%%%%%%%%%%%%%%%

%%%%%%%%%%%%%%%%%%%%%%%%%%%%%%%%%%%%%%
%     LINKS
%%%%%%%%%%%%%%%%%%%%%%%%%%%%%%%%%%%%%%

\section{Links}
Github:// \href{https://github.com/ColCross}{\custombold{ColCross}} \\
LinkedIn://  \href{https://www.linkedin.com/in/coltencross}{\custombold{ColtenCross}}
\sectionsep

%%%%%%%%%%%%%%%%%%%%%%%%%%%%%%%%%%%%%%
%     EDUCATION
%%%%%%%%%%%%%%%%%%%%%%%%%%%%%%%%%%%%%%

\section{Education} 
\descript{Georgia State University}
\location{May 2017 | Atlanta, GA}
\descript{BS in Physics}
\location{Comp Sci Concentration}
\sectionsep

%%%%%%%%%%%%%%%%%%%%%%%%%%%%%%%%%%%%%%
%
%     COLUMN TWO
%
%%%%%%%%%%%%%%%%%%%%%%%%%%%%%%%%%%%%%%

\end{minipage} 
\hfill
\begin{minipage}[t]{0.66\textwidth} 

%%%%%%%%%%%%%%%%%%%%%%%%%%%%%%%%%%%%%%
%     SKILLS
%%%%%%%%%%%%%%%%%%%%%%%%%%%%%%%%%%%%%%

\section{Skills}
\descript{Programming Languages}
NodeJS \textbullet{} JS \textbullet{} HTML/CSS \textbullet{} Git \textbullet{ C\# }\\
\descript{Frameworks}
ReactJS \textbullet{} Redux \textbullet{} Sagas \textbullet{} Serverless \textbullet{} RabbitMQ \textbullet{} ASP.NET\\
\descript{Cloud Services}
AWS - Lambda, SNS, S3, Parameter Store\\
\descript{Storage}
PostgreSQL \textbullet{} MongoDB \textbullet{} Elasticsearch \textbullet{} DynamoDB \textbullet{} Solr\\
\sectionsep

\end{minipage}
\sectionsep

%%%%%%%%%%%%%%%%%%%%%%%%%%%%%%%%%%%%%%
%     WORK EXPERIENCE
%%%%%%%%%%%%%%%%%%%%%%%%%%%%%%%%%%%%%%

\sectionsep
\section{Experience}
\runsubsection{CNN}
\descript{| Software Developer}
\location{Jan 2018 – Present | Atlanta, GA}
\begin{tightemize}
\item Developing front end features in ReactJS with saga middleware, and back end infrastructure with NodeJS, MongoDB, and PostgreSQL for the Newsource service. 
\item Implementing front end features in ReactJS for an embeddable video player for the Video Affiliate Network (VAN).
\item Writing and supporting cloud native services through AWS Lambda and the Serverless framework using Node, SNS, DynamoDB, and RabbitMQ.
\item Designing features and requirements alongside technical and functional team members, including architecture and testing strategies.
\item Collaborating directly with QA to find and resolve defects in features.
\item Employing full Agile methodologies through daily scrums, N+1's, grooming sessions, sprint retros, and Rally.
\item Providing 24/7 application support through an on-call rotation.
\item Developed minor features using ASP.NET, C\#, and Solr for the Resource application.
\end{tightemize}
\sectionsep

\runsubsection{Cartoon Network}
\descript{| Web Developer Intern}
\location{Aug 2017 – Dec 2017 | Atlanta, GA}
\begin{tightemize}
\item Used JavaScript and jQuery to convert Cartoon Network website login and registration functionality to a new internal system and API.
\item Collaborated with designers to implement front-end redesign for registration, login and profile pages using HTML5/CSS3.
\item Performed dev ops tasks such as updating miscellaneous pages, forms, and adding games.
\end{tightemize}
\sectionsep

\sectionsep
\section{Accomplishments}
\runsubsection{VAN Ingestion Pipeline}
\location{| CNN Newsource}
Transitioned the VAN video ingestion pipeline from a serial processing ASP.NET application to fault tolerant, parallel processing Node Lambdas. Assisted in determining requirements and designing architecture. Wrote a concurrency limited data migration script to move legacy videos into the new system.
\sectionsep

\runsubsection{VAN Video Player}
\location{| CNN Newsource}
Built front end animations, custom overlays, and a tooltip that dynamically adjusts based on the parent page. Minimal libraries were used in order to keep player file size small and quick to load.
\sectionsep

\end{document}  \documentclass[]{article}